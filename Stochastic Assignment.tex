\documentclass[a4paper,10pt]{article} % Uses article class in A4 format

%----------------------------------------------------------------------------------------
%	FORMATTING
%----------------------------------------------------------------------------------------

\setlength{\parskip}{0pt}
\setlength{\parindent}{0pt}
\setlength{\voffset}{-15pt}

%----------------------------------------------------------------------------------------
%	PACKAGES AND OTHER DOCUMENT CONFIGURATIONS
%----------------------------------------------------------------------------------------

\usepackage[a4paper, margin=2.5cm]{geometry} % Sets margin to 2.5cm for A4 Paper
\usepackage[onehalfspacing]{setspace} % Sets Spacing to 1.5
\usepackage{amsmath}
\usepackage{hyperref}
\usepackage{xcolor}
\usepackage{amsfonts}

\usepackage{titlesec} % Allows customization of titles
\renewcommand\thesection{\arabic{section}.} % Arabic numerals for the sections
\titleformat{\section}{\large}{\thesection}{1em}{}
\renewcommand\thesubsection{\alph{subsection})} % Alphabetic numerals for subsections
\titleformat{\subsection}{\large}{\thesubsection}{1em}{}
\renewcommand\thesubsubsection{\roman{subsubsection}.} % Roman numbering for subsubsections
\titleformat{\subsubsection}{\large}{\thesubsubsection}{1em}{}

\usepackage[all]{nowidow} % Removes widows

\usepackage[backend=biber,style=numeric,
sorting=nyt, natbib=true]{biblatex} % Complete reimplementation of bibliographic facilities
\addbibresource{main.bib}
\usepackage{csquotes} % Context sensitive quotation facilities

\usepackage[yyyymmdd]{datetime} % Uses YEAR-MONTH-DAY format for dates
\renewcommand{\dateseparator}{-} % Sets dateseparator to '-'

\usepackage{fancyhdr} % Headers and footers
\pagestyle{fancy} % All pages have headers and footers
\fancyhead{}\renewcommand{\headrulewidth}{0pt} % Blank out the default header
\fancyfoot[L]{\textsc{Emon Hossain, SH-019-120}} % Custom footer text
\fancyfoot[C]{} % Custom footer text
\fancyfoot[R]{\thepage} % Custom footer text

\newcommand{\note}[1]{\marginpar{\scriptsize \textcolor{red}{#1}}} % Enables comments in red on margin

%----------------------------------------------------------------------------------------

\begin{document}
	
	%----------------------------------------------------------------------------------------
	%	TITLE SECTION
	%----------------------------------------------------------------------------------------
	
	\title{template_assignment} % Article title
	\fancyhead[C]{}
	\begin{minipage}{0.295\textwidth} % Left side of title section
		\raggedright
		MTH-309\\ % Your lecture or course
		\footnotesize % Authors text size
		%\hfill\\ % Uncomment if right minipage has more lines
		Emon Hossain, Roll:SH-019-120 Reg:2017-613-811, Session:2019-20 (3rd year) % Your name, your matriculation number
		\medskip\hrule
	\end{minipage}
	\begin{minipage}{0.4\textwidth} % Center of title section
		\centering 
		\large % Title text size
		Assignment 01\\ % Assignment title and number
		\normalsize % Subtitle text size
		Stochastic Calculus\\ % Assignment subtitle
	\end{minipage}
	\begin{minipage}{0.295\textwidth} % Right side of title section
		\raggedleft
		\today\\ % Date
		\footnotesize % Email text size
		%\hfill\\ % Uncomment if left minipage has more lines
		mdemon7475@gmail.com% Your email
		\medskip\hrule
	\end{minipage}
	
	%----------------------------------------------------------------------------------------
	%	ARTICLE CONTENTS
	%----------------------------------------------------------------------------------------
	
	% here be dragons
	
	\vspace{10px}	

$5.1$ $\color{blue}{\textrm{Let }\Omega=\{a,b,c\}\textrm{ and let } X:\Omega\rightarrow \mathbb R^1 \textrm{ be a random variable defined by }X(a)=0,\quad X(b)=X(c)=1}$\\

$
\begin{aligned}
	(i)& \textrm{ What is the }\sigma-\textrm{algebra generated by }X?\\
	(ii)& \textrm{ If } Y\textrm{ is defined by }Y(a)=0,Y(b)=1,Y(c)=2,\textrm{ then what is }\mathbb E[X|Y]? \\
	(iii)&\textrm{ Prove or disprove }\mathbb E[X]=\mathbb E[\mathbb E[X|Y]]
\end{aligned}
$\\

\textbf{Answer	}:

$(i)$

Consider a random variable $X$. We know that $X$ is nothing but a measurable function from $\left(\Omega, \mathcal{A} \right)$ into $\left(\mathbb{R}, \mathcal{B}(\mathbb{R}) \right)$, where $\mathcal{B}(\mathbb{R})$ are the Borel sets of the real line. By definition of measurability we know that we have 

$$X^{-1} \left(B \right) \in \mathcal{A}, \quad \forall B \in \mathcal{B}\left(\mathbb{R}\right)$$

But in practice the preimages of the Borel sets may not be all of $\mathcal{A}$ but instead they may constitute a much coarser subset of it. To see this, let us define 

$$\Sigma = \left\{ S \in \mathcal{A}: S = X^{-1}(B), \  B \in \mathcal{B}(\mathbb{R}) \right\}$$

Using the properties of preimages, it is not too difficult to show that $\Sigma$ is a sigma-algebra.


Because $X$ has only two possible values $0$ and $1$ there are exactly four kinds of Borel sets $B$ relevant to $X$,

1. $1\in B$ and $0\in B.$  In this case, $X^{-1}(B) = \{\omega\in\Omega\mid X(\omega)\in B\}=\{a,b,c\}= \Omega.$

2. $1\in B$ but $0\notin B.$  Now $X^{-1}(B) = \{\omega\in\Omega\mid X(\omega)\in B\}=\{b,c\}.$

3. $1\notin B$ yet $0\in B.$  Now $X^{-1}(B) = \{\omega\in\Omega\mid X(\omega)\in B\}=\{a\}.$

4. $1\notin B$ and $0\notin B.$  Clearly $X^{-1}(B) = \emptyset.$

The $\sigma$-algebra generated by $X$ is $\mathcal F=\{\Omega,\{b,c\},\{a\},\emptyset\}$

$(ii)$

The $\sigma$-algebra generated by $Y$ is $\mathcal Y=\{\Omega,\{a\},\{b\},\{c\},\{a,b\},\{b,c\},\{a,c\},\emptyset\}$. And $X$ is $\mathcal Y$ measurable because $X^{-1}(\{0\})=\{a\}\in \mathcal Y$ and $X^{-1}(\{1\})=\{b,c\}\in \mathcal Y$. Hence, $$\mathbb E[X|Y]=X$$

$(iii)$

$$
\begin{aligned}
\mathbb E[\mathbb E[X|Y]]&=\int_{\Omega}\mathbb E[X|Y]\:dP \\
&=\int_{\Omega}X\:dP\\
&=\mathbb E[X]
\end{aligned}
$$\\

$5.2$ $\color{blue}{\textrm{Let }\Omega=\{a,b,c,d\}\textrm{ and let }\mathcal F\textrm{ be the }\sigma\textrm{-algebra consisting of all the subsets of }\Omega.\textrm{ And let }\mathcal G}$\\
	$\color{blue}{\textrm{ be the sub-}\sigma\textrm{-algebra generated by }\{a,b\}\textrm{ and }\{c,d\}}$\\

$
\begin{aligned}
	(i)& \textrm{List all of the measureable subsets belonging to }\mathcal G\\
	(ii)& \textrm{ Let }Z:\Omega\rightarrow\mathbb R^1 \textrm{ a function defined by }Z(a)=Z(b)=Z(c)=Z(d)=1.\\
	& \textrm{ Is }Z \textrm{ measurable with respect to }\mathcal G? \\
	(iii)&\textrm{ Suppose that }Y:\Omega\rightarrow\mathbb R^1\textrm{ is a }\mathcal G-\textrm{measurable function such that  }Y(a)=5.\\
	& \textrm{ What are possible value of }Y(b)?\\
	(iv)& \textrm{ Let }X:\Omega\rightarrow\mathbb R^1 \textrm{ be a random variable defined by }X(a)=0,X(b)=X(c)=3\textrm{ and }X(d)=1.\\
	& \textrm{ Let }\mathcal H\textrm{ be the sub-}\sigma\textrm{-algebra generated by }X.\textrm{ List all the elements of }\mathcal H.\\
	(v)&  \textrm{ Let }X\textrm{ be the random variable given in }(iv).\textrm{ Let }W:\Omega\rightarrow\mathbb R^1\textrm{ be a random variable defined by}\\
	& W(a)=10,W(b)=W(c)=W(d)=20.\textrm{ Find }\mathbb E[W|X].
\end{aligned}
$\\

\textbf{Answer	}:

$(i)$ $\mathcal G=\{\Omega,\{a,b\},\{c,d\},\emptyset\}$

$(ii)$ For a Borel subset $A \subset \mathbb R^1$, we have $Z^{-1}(A)=\Omega \in \mathcal G$ if $1 \in A$. Otherwise, $Z^{-1}(A)=\emptyset \in \mathcal G$. Hence $Z$ is $\mathcal G$-measurable.

$(iii)$ Since $Y^{-1}(\{5\})$ is $\mathcal G$-measurable and contains $a$, $Y^{-1}(\{a,b\})=\{a,b\}$ or $Y^{-1}(\{5\})=\Omega$. In either case, $Y(b)=5$.

$(iv)$ Since $\mathcal H$ contains $X^{-1}(\{0\})=\{a\},X^{-1}(\{3\})=\{b,c\},X^{-1}(\{1\})=\{d\}$, we have $\mathcal H=\sigma(X)=\{\Omega,\{a\},\{b,c\},\{d\},\{a,b,c\},\{a,d\},\{b,c,d\},\emptyset\}$

$(v)$ Note that $W$ is $\mathcal H$-measurable since $W^{-1}(\{10\})=\{a\}\in \mathcal H,W^{-1}(\{20\})=\{b,c,d\}\in \mathcal H$. Hence $\mathbb E[W|X]=W$.\\

$5.3$ $\color{blue}{\textrm{ Let }\Omega=\{uu,ud,du,dd\}\textrm{ and let }\mathcal F\textrm{ be the}}$ $\sigma$-$\color{blue}{\textrm{algebra consisting of the subsets of }\Omega}$. Let $X:\Omega\rightarrow\mathbb R^1$ $\color{blue}{\textrm{ be a random variable defined by }X(uu)=5,X(ud)=X(du)=3\textrm{ and }X(dd)=1}$.\\

$
\begin{aligned}
(i)& \textrm{ What is the sub-}\sigma\textrm{-algebra generated by }X?\\
(ii)& \textrm{ Let }Y:\Omega\rightarrow\mathbb R^1 \textrm{ be a random variable defined by }Y(uu)=1,Y(uu)=Y(du)=Y(ud)=-1.\textrm{ Find }\mathbb E[Y|X].\\
(iii)& \textrm{ Prove or disprove }(\mathbb E[Y|X])^2=\mathbb E[Y^2|X].
\end{aligned}
$


\textbf{Answer	}:\\
$(i)$ The sub-$\sigma$-algebra generated by $X$ is $\mathcal H=\{\Omega,\{uu\},\{ud,du\},\{dd\},\{uu,ud,du\},\{uu,dd\},\{ud,du,dd\},\emptyset\}$

$(ii)$ $Y$ is $\mathcal H$ measurable, hence $\mathbb E[Y|X]=Y$

$(iii)$ From $(ii)$ we get, $\mathbb E[Y|X]=Y$,
$$
\begin{aligned}
(\mathbb E[Y|X])^2&=Y^2\\
\mathbb E[Y^2|X]&=Y\mathbb E[Y|X]\quad\textrm{Pulling out known factors}\\
&=Y^2
\end{aligned}
$$\\

$5.4$ Let $\Omega=[0,1]$ with $\mathbb P$ $\color{blue}{\textrm{ the Lebesgue measure on }[0,1].\textrm{ Suppose that }X\textrm{ and }Y}$\\
$\color{blue}{\textrm{ are random variables on }(\Omega,\mathbb P),}$
$\color{blue}{\textrm{ and }Y(\omega)=\omega(1-\omega).\textrm{ Find }\mathbb E[X|Y].}$\\
$\color{blue}{\textrm{ (We assume that all the integrals under consideration exist.)}}$\\

\textbf{Answer	}:\\
Since $Y(\omega)=Y(1-\omega)$, letting the $\sigma$-algebra generated by $Y$ is $\mathcal F$ which consists of Borel sets $B\subset [0,1]$ such that $B=1-B$, where $1-B=\{1-x:x\in B\}$. For any such $B$,

$$
\begin{aligned}
\int_B X\: d\mathbb P &=\frac12 \int_B X(\omega)\: d\mathbb P+\frac12 \int_{1-B} X(1-\omega)\: d\mathbb P\\
&= \frac12 \int_B X(\omega)\: d\mathbb P+\frac12 \int_B X(1-\omega)\: d\mathbb P\\
&= \int_B \frac{X(\omega)+X(1-\omega)}{2}\: d\mathbb P
\end{aligned}
$$

Because $ \frac{X(\omega)+X(1-\omega)}{2}$ is $\sigma(Y)$ measurable, it follows that, $$\mathbb E[X|Y]=\frac{X(\omega)+X(1-\omega)}{2}$$\\

$5.7$ $\color{blue}{\textrm{ For two random variables }X\textrm{ and }Y,\textrm{ define the conditional variance of }X\textrm{ given }Y}$ by $$\textrm{Var} (X|Y)=\mathbb E[(X-\mathbb E[X|Y])^2|Y]$$ $\color{blue}{\textrm{show that}}$ $$\textrm{Var} (X|Y)=\mathbb E[X^2|Y]-\mathbb E[X|Y]^2$$ and $$\textrm{Var} (X|Y)=\mathbb E[\textrm{Var}(X|Y)]+\textrm{Var}(\mathbb E[X|Y])$$\\

\textbf{Answer	}: We have,

$$
\begin{aligned}
	\operatorname{Var}(X \mid Y) &=\mathbb{E}\left[(X-\mathbb{E}[X \mid Y])^{2} \mid Y\right] \\
	&=\mathbb{E}\left[X^{2} \mid Y\right]-2 \mathbb{E}[X \mid Y]^{2}+\mathbb{E}[X \mid Y]^{2} \\
	&=\mathbb{E}\left[X^{2} \mid Y\right]-\mathbb{E}[X \mid Y]^{2}
\end{aligned}
$$
Hence
$$
\mathbb{E}[\operatorname{Var}(X \mid Y)]=\mathbb{E}\left[\mathbb{E}\left[X^{2} \mid Y\right]\right]-\mathbb{E}\left[\mathbb{E}[X \mid Y]^{2}\right]=\mathbb{E}\left[X^{2}\right]-\mathbb{E}\left[\mathbb{E}[X \mid Y]^{2}\right] .
$$
On the other hand,
$$
\operatorname{Var}(\mathbb{E}[X \mid Y])=\mathbb{E}\left[\mathbb{E}[X \mid Y]^{2}\right]-\mathbb{E}[\mathbb{E}[X \mid Y]]^{2}=\mathbb{E}\left[\mathbb{E}[X \mid Y]^{2}\right]-\mathbb{E}[X]^{2}
$$
Therefore, $\mathbb{E}[\operatorname{Var}(X \mid Y)]+\operatorname{Var}(\mathbb{E}[X \mid Y])=\mathbb{E}\left[X^{2}\right]-\mathbb{E}[X]^{2} .$\\

% https://math.stackexchange.com/questions/1742578/law-of-total-variance-intuition
%$$\begin{aligned}
%	\mathsf {E}\big(\mathsf {Var} (Y\mid X)\big)
%	~=~& \mathsf E\big(\mathsf E(Y^2\mid X)\big)-\mathsf E\big(\mathsf E(Y\mid X)^2\big) 
%	\\[1ex] ~=~& \mathsf E(Y^2)-\mathsf E\big(\mathsf E(Y\mid X)^2\big)
%	\\[2ex]
%	\mathsf {Var}\big(\mathsf {E} (Y\mid X)\big) 
%	~=~& \mathsf E\big(\mathsf E(Y\mid X)^2\big)-\mathsf E\big(\mathsf E(Y\mid X)\big)^2
%	\\[1ex] ~=~& \mathsf E\big(\mathsf E(Y\mid X)^2\big)-\mathsf E(Y)^2
%	\\[2ex] \hline
%	\therefore ~ \mathsf {E}\big(\mathsf {Var} (Y\mid X)\big)+ \mathsf {Var}\big(\mathsf {E} (Y\mid X)\big)
%	~=~& \mathsf E(Y^2)-\mathsf E(Y)^2 
%	\\[1ex] ~=~& \mathsf {Var}(Y)
% end{aligned}$$

$6.2$ $\color{blue}{\textrm{ Let }Z_1,Z_2,\cdots,\textrm{ be independent and identically distributed random variables such that}}$ $\Pr(Z_n=1)=p>\frac12$ and $\Pr(Z_n=-1)=q=1-p$. Let $X_0,X_n=Z_1+Z_2+\cdots+Z_n,n\geq1$,$\color{blue}{\textrm{ be a random walk.}}$\\
$\color{blue}{\textrm{Prove that }X_n\textrm{ is not a martingale.}}$\\

	
\textbf{Answer	}:\\

$$
\begin{aligned}
\mathbb E[X_{n+1}|\mathcal F_n] &= \mathbb E[X_{n}+Z_{n+1}|\mathcal F_n]\\
&= \mathbb E[X_{n}|\mathcal F_n]+\mathbb E[Z_{n+1}|\mathcal F_n]\\
&= X_n+p-(1-p)\\
&= X_n+\underbrace{(2p-1)}_{>0}
\end{aligned}
$$

Which is not martingale.


$6.3$ $\color{blue}{\text{ Let }M_0,M_1,\cdots\textrm{ be a martingale. Show that }\exp(M_0),\exp(M_1),cdots\textrm{ is a submartingale.}}$\\

\textbf{Answer	}:\\
$$
\begin{aligned}
\mathbb E[\exp(M_{n+1})|\mathcal F_n]&\geq \exp(\mathbb E[M_{n+1}|\mathcal F_n])\quad\textrm{From Jensen's inequality}\\
&= \exp(M_n) \quad\textrm{ As }(M_n)_{n\geq0}\textrm{ is martingale}
\end{aligned}
$$	

Hence, $(\exp(M_{n}))_{n\geq0}$ is a submartingale.\\

$6.4$ $\color{blue}{\text{ Show that a discrete time previsible martingale is constant.}}$\\ 

Suppose that $(X_{n})_{n\geq 0}$ is a previsible martingale. We have to take advantage of each property :

First, since it is a martingale, we have:
$$\mathbb{E}[X_{n+1}|\mathcal{F}_{n}]=X_{n}$$

and since it is predictable, we know that $X_{n+1}$ is $\mathcal{F}_{n}$ measurable, so
$$\mathbb{E}[X_{n+1}|\mathcal{F}_{n}]=X_{n+1}$$

The above two results yield
\begin{equation}
	X_{n}=X_{n+1}
\end{equation}

The proof is an immediate consequence.\\

$6.10$ $\color{blue}{\text{Assume that }\mathcal{F}_0=\{\emptyset,\Omega\}.\text{ Show that if a discrete time martingale }\{X_n\}\text{ is predictable, then it is constant.}}$\\

\textbf{Answer	}:\\

Since  $X_{n-1}\stackrel{\textrm{martingale}}{=}\mathbb E[X_n|\mathbb F_{n-1}]\stackrel{\textrm{predictable}}{=}X_n$ for every $n$, we have $$X_n=X_{n-1}=\cdots=X_1=X_0=\mathbb E[X_1|\mathcal F_0]$$ and $X_n$ is constant.


\end{document}